\documentclass{article}
\usepackage{graphicx}
\usepackage{tikz}
\usetikzlibrary{positioning,arrows.meta,spy}

\definecolor{arrowblue}{RGB}{98,145,224}

\newcommand\ImageNode[3][]{
  \node[draw=arrowblue!80!black,line width=1pt,#1] (#2) {\includegraphics[width=1.5cm,height=1.5cm]{#3}};
  }

\begin{document}
\begin{tikzpicture}[
  spy using outlines={
    magnification=4, 
    size=1.5cm, 
    connect spies
  },
  node distance=1.5cm,
  >={Triangle[angle=60:1pt 2]},
  shorten >= 2pt,
  shorten <= 2pt,
  arrow/.style={
    ->,
    arrowblue,
    line width=5pt
  },
  aux/.style={
    text width=1.5cm,
    minimum height=1.75cm,
    draw=arrowblue!80!black,
    line width=1pt
  }
]

\ImageNode[label={-90:Normal Image Sample}]{A}{img1.png}
\ImageNode[right=of A]{B}{deepface-icon-labeled.png}
\ImageNode[label={-90:Detected}, above right=of B]{C}{img1_detect.png}
\ImageNode[label={0:Probabilities for Emotion}, below right=of B]{D}{prob_final.png}
\ImageNode[label={-90:Sklearn Model},below=of D]{E}{ML.png}
\ImageNode[label={-90:Range for Normal},right=of E]{H}{range.png}
\ImageNode[label={-90:Potential Patients},below left=of E]{F}{img3.png}
\ImageNode[below right=of E]{G}{deepface-icon-labeled.png}
\ImageNode[label={0:Probabilities for Emotion}, below =of G]{K}{prob.png}
\ImageNode[label={-90:Parkinson Disease},below left=of K]{I}{img3.png}
\ImageNode[label={-90:Not Parkinson Disease},below right=of K]{J}{img3.png}

\draw[arrow]
  (A) -- node[above,text=black] {Input} (B);

\draw[arrow]
  (B)--(C);

\draw[arrow]
  (B)--node[above right,text=black]{output}(D);

\draw[arrow]
  (D)--node[right, text=black]{Tranning}(E);

\draw[arrow]
  (F)--node[above, text=black]{Input}(G);

\draw[arrow]
  (E)--node[above, text=black]{Output}(H);

\draw[arrow]
  (K)--node[above left, text=black]{Out of Range}(I);

\draw[arrow]
  (K)--node[above right, text=black]{In Range}(J);

\draw[arrow]
  (G)--node[right, text=black]{Output}(K); 

\end{tikzpicture}


\end{document}